# EstCosmoPar-GPR-TPR
Estimation of Cosmological Priors using Gaussian Processes and Viable Alternatives

Several papers within the astrophysical literature are dedicated to obtaining accurate and reliable estimates for the Hubble constant $H_0$, using diverse data sources such as CC, SNIa, and BAO, and different methodologies. This results in estimates which do not agree with each other - the so-called `$H_0$ tension'. In this work, methods already established in literature for estimating $H_0$, such as Gaussian process regression (GPR) and Markov chain Monte Carlo (MCMC) methods based on the concordance Lambda Cold Dark Matter ($\Lambda CDM$) model, together with some novel approaches in the field, are assessed. The first novel approach makes use of non-parametric MCMC inference on the hyperparameters of a Gaussian process kernel, independently of any cosmological model such as $\Lambda CDM$. The second approach is Student's $t$-process regression (TPR), which is similar to GPR but makes use of the multivariate Student's $t$-distribution instead of the multivariate Gaussian distribution. TPR does not automatically assume Gaussianity of underlying observations and has the additional advantage of being a more generalised and flexible form of GPR. We also consider variants of GPR and TPR which account for heteroscedasticity within the data. A comparison of the novel and tried-and-tested approaches is made. In particular, the model-independent approaches investigated largely agree with predictions based on the $\Lambda CDM$ concordance model. Moreover, GPR is highly dependent on the prior specification, while TPR and the heteroscedastic variants of both GPR and TPR are more robust to this. TPR and both heteroscedastic models provide evidence for a Hubble constant value that is on the lower side, and is therefore closer to the Planck value of $67.4$ km s\textsuperscript{-1} Mpc\textsuperscript{-1} \ than the Riess value of $74.22$ km s\textsuperscript{-1} Mpc\textsuperscript{-1}. Therefore, the novel approaches discussed in this dissertation may shed further light on the $H_0$ tension. A main challenge posed by these approaches is the use of small datasets for the model-independent approaches; further research can thus apply such approaches to larger datasets. Across all estimates obtained for $H_0$ in this work, the median value is $\hat{H}_0^{med} = 68.85 \pm 1.67$ km s\textsuperscript{-1} Mpc\textsuperscript{-1}.
